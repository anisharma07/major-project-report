\chapter{Introduction}
% Line
\noindent\rule{\linewidth}{2pt} 

\section{Overview}
As urban populations grow, efficient mobility becomes a paramount challenge. Parking management is a crucial aspect of this, yet it remains largely inefficient in many cities. \textbf{ParkLot} aims to solve this by introducing a smart detection system that eliminates the need for drivers to circle endlessly looking for spots.

\section{Motivation}
The motivation for this project stems from three key factors:
\begin{itemize}
    \item \textbf{Traffic Reduction:} Approximately 30\% of urban traffic is attributed to drivers searching for parking. Smart detection can significantly alleviate this congestion.
    \item \textbf{Environmental Impact:} Reducing the time spent searching for parking directly lowers fuel consumption and the associated carbon footprint.
    \item \textbf{Cost Efficiency:} Traditional smart parking relies on expensive individual sensors for each spot. Our approach unlocks intelligence from existing CCTV infrastructure, removing the need for new sensor hardware.
\end{itemize}

\section{Problem Statement}
The fundamental goal of this project is to achieve highly effective parking space detection using computer vision. The specific challenges addressed include:
\begin{itemize}
    \item \textbf{Aerial View Occlusions:} The system must process aerial datasets which provide a wide view but are susceptible to occlusions where larger vehicles may block the view of empty spots.
    \item \textbf{Model Selection:} Determining which version of the YOLO family (v5, v7, v8, v9) offers the optimal balance between accuracy and real-time performance for this specific domain.
\end{itemize}

\section{Objectives}
The primary objectives of the ParkLot project are:

\begin{enumerate}
    \item Develop a vision-based vehicle detection system using YOLO framework capable of accurately identifying vehicle presence and determining parking slot occupancy in real-time.
    
    \item Create an intelligent parking management platform that processes CCTV feeds to provide live parking availability information across designated parking zones.
    
    \item Design and implement a mobile application interface that enables users to discover nearby parking facilities, view real-time availability, and navigate to vacant spots.
    
    \item Deploy a proof-of-concept implementation at NIT Hamirpur campus to validate system performance in real-world conditions including outdoor environments and irregular parking layouts.
    
    \item Optimize the system for edge computing deployment (Raspberry Pi) to minimize latency and ensure functionality with intermittent internet connectivity.
    
    \item Demonstrate measurable improvements in parking detection accuracy and inference speed suitable for real-time applications.
\end{enumerate}

\section{Report Organization}

The remainder of this report is organized as follows:

\textbf{Chapter 2: Literature Review} examines existing research in parking detection, YOLO architecture evolution, and comparison of different approaches.

\textbf{Chapter 3: System Design and Architecture} details the overall system design, hardware specifications, cloud infrastructure, and component interactions.

\textbf{Chapter 4: Methodology} describes dataset preparation, model training procedures, and fine-tuning strategies.

\textbf{Chapter 5: Results and Analysis} presents performance metrics, accuracy analysis, and comparative evaluation of different YOLO models.

\textbf{Chapter 6: Conclusion and Future Work} summarizes achievements, discusses system capabilities, and outlines future enhancements.