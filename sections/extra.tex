\newpage
\thispagestyle{empty}
\cleardoublepage
\phantomsection

% Way to add things to TOC
\addcontentsline{toc}{chapter}{Appendix}

\chapter*{Appendix: Code Snippets and Configuration}

\section{Edge Device Configuration}

The following JSON configuration file is used to configure each Raspberry Pi edge device:

\begin{verbatim}
{
    "camera_type": "web_upload",
    "api_endpoint": "http://34.42.200.32/parking/updateRaw",
    "camera_id": "web_upload_camera",
    "interval": 60,
    "save_local_copy": true,
    "model_path": "models/yolov8n_custom.pt",
    "confidence_threshold": 0.5,
    "device": "cpu",
    "half_precision": false
}
\end{verbatim}

\section{Model Training Script (Python)}

\begin{verbatim}
from ultralytics import YOLO

# Load pretrained model
model = YOLO('yolov8n.pt')

# Fine-tune on custom dataset
results = model.train(
    data='data/data.yaml',
    epochs=50,
    imgsz=416,
    batch=16,
    device=0,
    optimizer='SGD',
    patience=10,
    save=True,
    project='runs/parking'
)

# Evaluate on validation set
metrics = model.val()
\end{verbatim}

\section{Inference Pipeline (Python)}

\begin{verbatim}
import cv2
from ultralytics import YOLO

# Load trained model
model = YOLO('models/yolov8n_parking.pt')

# Process video stream
cap = cv2.VideoCapture(0)

while True:
    ret, frame = cap.read()
    if not ret:
        break
    
    # Resize for inference
    frame_resized = cv2.resize(frame, (416, 416))
    
    # Run inference
    results = model(frame_resized, conf=0.5)
    
    # Process detections
    for det in results[0].boxes.xyxy:
        # Extract bounding box coordinates
        x1, y1, x2, y2 = det.numpy()
        cv2.rectangle(frame, (x1, y1), (x2, y2), (0, 255, 0), 2)
    
    # Display results
    cv2.imshow('Parking Detection', frame)
    if cv2.waitKey(1) & 0xFF == ord('q'):
        break

cap.release()
cv2.destroyAllWindows()
\end{verbatim}

\section{API Endpoint Example (Flask)}

\begin{verbatim}
from flask import Flask, request, jsonify
from pymongo import MongoClient

app = Flask(__name__)
db = MongoClient('mongodb://localhost:27017/')['parking']

@app.route('/parking/updateRaw', methods=['POST'])
def update_parking_raw():
    """Update parking occupancy from edge device"""
    data = request.json
    
    # Validate data
    if 'camera_id' not in data or 'available_spots' not in data:
        return jsonify({'error': 'Missing fields'}), 400
    
    # Update database
    db.parking_status.update_one(
        {'camera_id': data['camera_id']},
        {
            '\$set': {
                'available_spots': data['available_spots'],
                'total_spots': data['total_spots'],
                'timestamp': datetime.datetime.utcnow()
            }
        },
        upsert=True
    )
    
    return jsonify({'status': 'success'})

@app.route('/parking/nearby', methods=['GET'])
def get_nearby_parking():
    """Get nearby parking locations"""
    lat = float(request.args.get('lat'))
    lon = float(request.args.get('lon'))
    radius = float(request.args.get('radius', 10))
    
    # Query nearby locations
    locations = db.parking_location.find({
        'location': {
            '\$near': {
                '\$geometry': {
                    'type': 'Point',
                    'coordinates': [lon, lat]
                },
                '\$maxDistance': radius * 1000
            }
        }
    })
    
    return jsonify(list(locations))

if __name__ == '__main__':
    app.run(host='0.0.0.0', port=5000)
\end{verbatim}

\section{Dataset Configuration (data.yaml)}

\begin{verbatim}
path: /path/to/parking_dataset
train: images/train
val: images/val
test: images/test

nc: 1
names: ['Car']

# Optional parameters
download: false
\end{verbatim}

\section{Docker Configuration}

\begin{verbatim}
FROM python:3.10-slim

WORKDIR /app

COPY requirements.txt .
RUN pip install -r requirements.txt

COPY . .

EXPOSE 5000

CMD ["python", "app.py"]
\end{verbatim}

\section{Model Performance Logging}

\begin{verbatim}
import json
import logging

# Setup logging
logging.basicConfig(
    filename='parking_inference.log',
    level=logging.INFO,
    format='%(asctime)s - %(levelname)s - %(message)s'
)

# Log inference results
def log_inference(camera_id, inference_time, detections):
    log_entry = {
        'camera_id': camera_id,
        'inference_time_ms': inference_time,
        'num_vehicles_detected': len(detections),
        'timestamp': datetime.datetime.utcnow().isoformat()
    }
    logging.info(json.dumps(log_entry))
\end{verbatim}